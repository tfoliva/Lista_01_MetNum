\documentclass{article}
\usepackage[brazil]{babel}
\usepackage[T1]{fontenc}
\usepackage{inputenc}
\usepackage{enumitem}
\usepackage{amsmath}
\usepackage{amssymb}
\usepackage{amsfonts}
%\usepackage[pdftex]{graphicx}
%\usepackage{subfigure}

\newcommand{\euler}{\mathrm{e}}
\title{Métodos Numéricos --- Lista 01} 
\author{André Paladini  \quad 14182390 \\ Tiago F. Oliva Costa \quad 8004408 }

\begin{document}
\maketitle

\section{Questão 01}
Classifique as EDPs abaixo quanto à ordem, a linearidade / não-linearidade, a homogeneidade e ao tipo.

\begin{enumerate}[label=\Alph*]
\item 2a ordem; Linear; Homogênea.
\item 2a ordem; Linear; Não-Homogênea.
\item 1a ordem; Linear; Homogênea.
\item 2a ordem; Linear; Homogênea.
\item 2a ordem; Não-Linear; Homogênea.
\item 2a ordem; Não-Linear; Não-Homogênea.
\end{enumerate}

\section{Questão 02}
Qual a diferença entre as condições de contorno de Dirichlet, Neumann e Robin?

A condição de contorno de Dirichlet (ou primeiro tipo) especifica valores que a variável dependente $y(x)$ toma ao longo da fronteira do domínio. Ou seja
\[ y(a) = \alpha, \quad y(b) = \beta. \]

A condição de contorno de Neumann (ou segundo tipo) especifica valores que a derivada $y'(x)$ da variável dependente  toma ao longo da fronteira do domínio. Ou seja
\[ y'(a) = \alpha, \quad y'(b) = \beta. \]

A condição de contorno de Robin (ou terceiro tipo) especifica valores que tanto a variável dependente $y(x)$, como a sua derivada $y'(x)$, tomam ao longo da fronteira do domínio. Ou seja, para um domínio $\Omega$ e sua fronteira representada por $\partial \Omega$, têm-se
\[ a y + b \frac{\partial y}{\partial x} =g \qquad \text{em } \quad \partial \Omega.\]

\section{Suplemento}
Para as questões 03 e 04, considere como \emph{forward difference}
\[
	\Delta_h f(x) = f(x+h) - f(x),
\]
\emph{central difference}
\[
	\delta_h f(x) = f(x+\frac{h}{2}) - f(x - \frac{x}{2}),
\]
e \emph{backward difference}
\[
	\nabla_h f(x) = f(x) - f(x-h).
\]

\section{Questão 03}

Pede-se $\frac{\mathrm{d}J_0(x)}{\mathrm{d}x}$ em $x=3$, onde 
\[J_\alpha(x) = \sum_{m=0}^\infty \frac{(-1)^m}{m!\, (m+\alpha)!} {\left(\frac{x}{2}\right)}^{2m+\alpha},\]
e por sua vez
\[J_0(x) = \sum_{m=0}^\infty \frac{(-1)^m}{m!\, m!} {\left(\frac{x}{2}\right)}^{2m}.\]

Considerando que a função de Bessel converge, podemos aplicar a derivada da série infinita obtendo
\[J_0'(x) = \sum_{m=0}^\infty \frac{(-1)^m}{m!\, (m-1)!} {\left(\frac{x}{2}\right)}^{2m-1} = J_{(-1)}(x) = -J_1(x).\]

Dessa forma temos 
(a) Backward com 2 pontos
(b) Backward com 3 pontos
(c) Forward com 2 pontos
(d) Forward com 3 pontos
(e) Central com 2 pontos
(f) Central com 4 pontos

\section{Questão 04}
Considere a função 
\[ f(x) = \euler^x \sin(x).
\]
Temos, pela regra do produto, 
\[ f'(x) = \euler^x \sin(x) + \euler^x\cos(x),
\]
e aplicando a regra do produto novamente
\[ f''(x) =  2\euler^x\cos(x).
\]

(i) Central com 2 pontos
(ii) Central com 4 pontos


\end{document}
