\documentclass{article}
\usepackage[brazil]{babel}
\usepackage[T1]{fontenc}
\usepackage{inputenc}
\usepackage{enumitem}
\usepackage{amsmath}
%\usepackage[pdftex]{graphicx}
%\usepackage{subfigure}

\title{Métodos Numéricos --- Lista 01} 
\author{André Paladini  \quad 14182390 \\ Tiago F. Oliva Costa \quad 8004408 }

\begin{document}
\maketitle

\section{Questão 01}
Classifique as EDPs abaixo quanto à ordem, a linearidade / não-linearidade, a homogeneidade e ao tipo.

\begin{enumerate}[label=\Alph*]
\item 2a ordem; Linear; Homogênea.
\item 2a ordem; Linear; Não-Homogênea.
\item 1a ordem; Linear; Homogênea.
\item 2a ordem; Linear; Homogênea.
\item 2a ordem; Não-Linear; Homogênea.
\item 2a ordem; Não-Linear; Não-Homogênea.
\end{enumerate}

\section{Questão 02}
Qual a diferença entre as condições de contorno de Dirichlet, Neumann e Robin?

A condição de contorno de Dirichlet (ou primeiro tipo) especifica valores que a variável dependente $y(x)$ toma ao longo da fronteira do domínio. Ou seja
\[ y(a) = \alpha, \quad y(b) = \beta. \]

A condição de contorno de Neumann (ou segundo tipo) especifica valores que a derivada $y'(x)$ da variável dependente  toma ao longo da fronteira do domínio. Ou seja
\[ y'(a) = \alpha, \quad y'(b) = \beta. \]

A condição de contorno de Robin (ou terceiro tipo) especifica valores que tanto a variável dependente $y(x)$, como a sua derivada $y'(x)$, tomam ao longo da fronteira do domínio. Ou seja, seja um domínio $\Omega$ e sua fronteira representada por $\partial \Omega$,
\[ a y + b \frac{\partial y}{\partial x} =g \qquad \text{em } \quad \partial \Omega.\]

\end{document}
